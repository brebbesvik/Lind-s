\documentclass{beamer}

\usepackage{hyperref}
\usepackage{graphicx}
\graphicspath{ {images/} }
\usepackage{tabularx}

\usetheme{Madrid}
\usecolortheme{default}
\setbeamertemplate{navigation symbols}{}

\begin{document}
	
	\begin{frame}{Gamification of Clinical Practice Guidelines - 1}
	\begin{itemize}
		\item A learning tool to train health workers and medical students.
		\item So what is Clinical Practice Guidelines?
	\end{itemize}
\end{frame}

\begin{frame}{Introduction - 2}
CPGS formalizes research, and is validated for use in treatment.
\begin{itemize}
	\item 
	\item We are weighing benefits and harm to each other.
	\item All patients gets the same treatment for the same condition.
	\item Since formalized, better use of resources, use cheaper medicine, work more efficient, higher quality.
\end{itemize}
\end{frame}

\begin{frame}{Introduction - 3}
\begin{itemize}
	\item  Lack of self-efficacy: the don't believe in themselves that they execute the instructions in the guideline with high quality. 
	\item +++: Lack of agreement with the content, Lack of outcome expectancy, Inertia of previous practice. 
\end{itemize}
\end{frame}

\begin{frame}{Norwegian Guideline - 4}
\begin{itemize}
	\item Large Guidelines. 400 pages.
	\item This one is used for emergency clinics in Norway.
	\item Wall of text
	\item Symptoms left, treatment right.
	\item Not good at point of care. We can improve.
\end{itemize}
\end{frame}

\begin{frame}{Research questions - 5}
\begin{itemize}
	\item Adaptable to the learner.
\end{itemize}
\end{frame}

\begin{frame}{Approach - 6}
I have done iterations and evaluations.
\end{frame}


\begin{frame}{Gamification of Clinical Practice Guidelines - 7}
\begin{itemize}
	\item Multiple-try with feedback. Multiple attempts to get a question right and the feedback is right or wrong.
	\item Problem with mulitple-try  with feedback: bad or unmotivated students will click randomly.
	\item Solution: a button where you ccan ontinue without selecting the correct answer.
	\item Small penalty and large reward motivates to revise an incorrect answer.
\end{itemize}
\end{frame}

\begin{frame}{Possible asthma in paediatrics - 8}
\begin{itemize}
	\item Assessment, get an idea of what diagnosis a patient may have.
	\item Diagnosis, strengthen your assumption and pick a severity.
	\item Management, treatment and procedure.
	\item Evaluation and act on the evaluation.
	\item What the patient shoud do on his own and follow-up.
\end{itemize}
\end{frame}

\begin{frame}{Workflow graph - 9}
\begin{itemize}
	\item It's the flow of the clinical encounter.
	\item Based on the guideline.
\end{itemize}
\end{frame}

\begin{frame}{Entity graph - 10}
\begin{itemize}
	\item This is a skeleton, an excerpt to keep the complexity down and keep it more compact.
	\item It represents the patient at a given time in the clinical encounter.
	\item You give the patient medication, and hopefully the symptoms in the graph has changed when you evaluate.
\end{itemize}
\end{frame}

\begin{frame}{Making scenarios - 11}
\begin{itemize}
	\item Here we show the detail level of the entity graph (patient).
	\item We are using templates to create narratives.
	\item Tags which refer to vertices in the graph. Both in template and answer key.
	\item Can use the same template with other graphs to make new narratives with matching answer key.
	\item Can also reuse the graph on other templates.
	\item Problem: The values can't be directly printed in the narrative. What is 24? What is True? V?
\end{itemize}
\end{frame}

\begin{frame}{Making scenarios - 12}
\begin{itemize}
\item Attach a presentation vertex to all values.
\item If a vertex doesn't have a presentation, simply return the value.
\end{itemize}
\end{frame}

\begin{frame}{Dynamic Content Management - 13}
\begin{itemize}
	\item A way to control the content flow in the application.
	\item Instead of every student going through the same learning material in the same way, we try to adapt it to the individual learner.
\end{itemize}
\end{frame}

\begin{frame}{Dynamic Content Management - 14}
\begin{itemize}
	\item We split the content in relation to the clinical encounter (workflow model).
\end{itemize}
\end{frame}

\begin{frame}{Dynamic Content Management - 15}
We have identified
\begin{itemize}
	\item  the knowledge units.
	\item the dependency between the knowledge units.
	\item expressed the dependency as a prerequisite.
\end{itemize}
Need to learn how to diagnise and treat a patient, before you can evaluate the treatment and act.
\end{frame}

\begin{frame}{Dynamic Content Management - 16}
\begin{itemize}
	\item The student can finish assessment, diagnosis or management in any order he wants.
	\item Have to complete all categories at one level to be able to progress to the next level.
	\item The detail level of the questions increases for each level.
	\item Demonstrate learning map and student map. 
\end{itemize}
\end{frame}

\begin{frame}{Demonstration - 17}
\begin{itemize}
	\item Pick a quiz.
	\item The difficulties shows where the student is in the learning map.
\end{itemize}
\end{frame}

\begin{frame}{Demonstration - 18}
\begin{itemize}
	\item Scenario.
	\item Assessment.
	\item Multiple choice.
	\item Feedback.
	\item Shows an explanation when correct.
	\item Wheeze.
\end{itemize}
\end{frame}

\begin{frame}{Demonstration - 19}
\begin{itemize}
	\item The scenario continues.
	\item Diagnosis.
	\item Oxygen saturation <90\%
\end{itemize}
\end{frame}

\begin{frame}{Demonstration - 20}
\begin{itemize}
	\item Management.
	\item Prednisolone and oxygen.
\end{itemize}
\end{frame}

\begin{frame}{Demonstration - 21}
\begin{itemize}
	\item Follow-up. Asks about prednisolone.
	\item Can try again. Gets penalty for every wrong attempt.
	\item Can choose to give up, see the answer and continue, but looses the reward.
	\item We want the user to encourage the user to revise the question.
	\item 5 days.
\end{itemize}
\end{frame}

\begin{frame}{Demonstration - 22}
\begin{itemize}
	\item We did very well and our position in the learning map is updated.
	\item We still have to redo follow-up before we can continue to level 3.
	\item The graph shows the score in each individual category as bars. The line is the passing condition in each category.
	\item We didn't pass follow-up.
\end{itemize}
\end{frame}

\begin{frame}{Evaluation - 23}
The evaluation needs to be scheduled.

Will do whatever we have time for.
\end{frame}
\end{document}